The \textit{Louvain} method is a greedy modularity-optimization based community detection algorithm, and is introduced by Blondel et al. from the University of Louvain \cite{com-blondel08}. It identifies communities with resulting high modularity, and is thus widely favored \cite{com-lancichinetti09}. Algorithmic improvements proposed for the original algorithm include early pruning of non-promising candidates (leaf vertices) \cite{com-ryu16, com-halappanavar17, com-zhang21, com-you22}, attempting local move only on likely vertices \cite{com-ryu16, com-ozaki16, com-zhang21, com-shi21}, ordering of vertices based on node importance \cite{com-aldabobi22}, moving nodes to a random neighbor community \cite{com-traag15}, threshold scaling \cite{com-lu15, com-naim17, com-halappanavar17}, threshold cycling \cite{com-ghosh18}, subnetwork refinement \cite{com-waltman13, com-traag19}, multilevel refinement \cite{com-rotta11, com-gach14, com-shi21}, and early termination \cite{com-ghosh18}.

To parallelize the Louvain algorithm, a number of strategies have been attempted. These include using heuristics to break the sequential barrier \cite{com-lu15}, ordering vertices via graph coloring \cite{com-halappanavar17}, performing iterations asynchronously \cite{com-que15, com-shi21}, using adaptive parallel thread assignment \cite{com-fazlali17, com-naim17, com-sattar19, com-mohammadi20}, parallelizing the costly first iteration \cite{com-wickramaarachchi14}, using vector based hashtables \cite{com-halappanavar17}, and using sort-reduce instead of hashing \cite{com-cheong13}\ignore{, using simple partitions based of vertex ids \cite{com-cheong13, com-ghosh18}, and identifying and moving ghost/doubtful vertices \cite{com-zeng15, com-que15, com-bhowmik19, com-bhowmick22}}. Platforms used range from an AMD multicore system \cite{com-fazlali17}, and Intel’s Knight's Landing, Haswell \cite{com-gheibi20}, SkylakeX, and Cascade Lake \cite{part-hossain21}. Other approaches include the use of MapReduce in a BigData batch processing framework \cite{com-zeitz17}. It should however be noted though that community detection methods such as the Louvain that rely on modularity maximization are known to suffer from resolution limit problem. This prevents identification of communities of certain sizes \cite{com-ghosh19}.

A few open source implementations and software packages have been developed for community detection. Vite \cite{ghosh2018scalable} is a distributed memory parallel implementation of the Louvain method that incorporates several heuristics to enhance performance while maintaining solution quality, while Grappolo \cite{com-halappanavar17} is a shared memory parallel implementation. NetworKit \cite{staudt2016networkit} is a software package designed for analyzing the structural aspects of graph data sets with billions of connections. It is implemented as a hybrid with C++ kernels and a Python frontend, and includes a parallel implementation of the Louvain algorithm.
