The Louvain method, introduced by Blondel et al. \cite{com-blondel08} from the University of Louvain, is a greedy modularity-optimization based algorithm for community detection \cite{com-lancichinetti09}. While it is favored for identifying communities with high modularity, it often results in internally disconnected communities. This occurs when a vertex, acting as a bridge, moves to another community during iterations. Further iterations aggravate the problem, without decreasing the quality function. Further, the Louvain method may identify communities that are not well connected, i.e., splitting certain communities could improve the quality score --- such as modularity \cite{com-traag19}.\ignore{This is not the same as resolution limit problem with modularity, that causes small communities to be clustered with large communities. Louvain only guarantees that no communities can be merged (well separated).}

To address these limitations, Traag et al. \cite{com-traag19} from the University of Leiden, propose the Leiden algorithm\ignore{as an enhancement of the Louvain method}. It introduces a \textit{refinement phase} after the local-moving phase, where vertices within each community undergo additional local moves in a randomized fashion proportional to the delta-modularity of the move. This allows vertices to find sub-communities within those obtained from the local-moving phase. The Leiden algorithm guarantees that the identified communities are both well separated (like the Louvain method) and well connected. When communities have converged, it is guaranteed that all vertices are optimally assigned, and all communities are subset optimal \cite{com-traag19}. Shi et al. \cite{com-shi21} also introduce an additional refinement phase after the local-moving phase with the Louvain method, which they observe minimizes bad clusters. It should however be noted that methods relying on modularity maximization are known to suffer from resolution limit problem, which prevents identification of communities of certain sizes \cite{com-ghosh19, com-traag19}. This can be overcome by using an alternative quality function, such as the Constant Potts Model (CPM) \cite{com-traag11}.

We now discuss a number of algorithmic improvements proposed for the Louvain method, that also apply to the Leiden algorithm. These include ordering of vertices based on importance \cite{com-aldabobi22}, attempting local move only on likely vertices \cite{com-ryu16, com-ozaki16, com-zhang21, com-shi21}, early pruning of non-promising candidates\ignore{(leaf vertices)} \cite{com-ryu16, com-halappanavar17, com-zhang21, com-you22}, moving vertices to a random neighbor community \cite{com-traag15}, subnetwork refinement \cite{com-waltman13, com-traag19}, multilevel refinement \cite{com-rotta11, com-gach14, com-shi21}, threshold cycling \cite{com-ghosh18}, threshold scaling \cite{com-lu15, com-naim17, com-halappanavar17}, and early termination \cite{com-ghosh18}.

A number of parallelization techniques have been attempted for the Louvain method, that may also be applied to the Leiden algorithm. These include\ignore{using heuristics to break the sequential barrier \cite{com-lu15},} using adaptive parallel thread assignment \cite{com-fazlali17, com-naim17, com-sattar19, com-mohammadi20}, parallelizing the costly first iteration \cite{com-wickramaarachchi14}, ordering vertices via graph coloring \cite{com-halappanavar17}, performing iterations asynchronously \cite{com-que15, com-shi21}, using vector based hashtables \cite{com-halappanavar17}, and using sort-reduce instead of hashing \cite{com-cheong13}\ignore{, using simple partitions based of vertex ids \cite{com-cheong13, com-ghosh18}, and identifying and moving ghost/doubtful vertices \cite{com-zeng15, com-que15, com-bhowmik19, com-bhowmick22}}.\ignore{Platforms used range from an AMD multicore system \cite{com-fazlali17}, and Intel’s Knight's Landing, Haswell \cite{com-gheibi20}, SkylakeX, and Cascade Lake \cite{part-hossain21}. Other approaches include the use of MapReduce in a BigData batch processing framework \cite{com-zeitz17}.}

A few open source implementations and software packages have been developed for community detection using Leiden algorithm. The original implementation of the Leiden algorithm \cite{com-traag19}, called \texttt{libleidenalg}, is written in C++ and has a Python interface called \texttt{leidenalg}. NetworKit \cite{staudt2016networkit} is a software package designed for analyzing the structural aspects of graph data sets with billions of connections. It utilizes a hybrid with C++ kernels and a Python frontend. The package features a parallel implementation of the Leiden algorithm by Nguyen \cite{nguyenleiden} which uses global queues for vertex pruning, and vertex and community locking for updating communities. igraph \cite{csardi2006igraph} is a similar package, written in C, with Python, R, and Mathematica frontends. It is widely used in academic research, and includes an implementation of the Leiden algorithm.
% I see two thesis by Fabian, Karlsruhe Institute of Technology and Magnus, University of Bergen. Fabian uses global queues and for vertex pruning, and vertex and community locking for updating communities, which is likely to be less efficient. Magnus uses the same approach as Fabien. On europe graph their algorithm takes 60-70s with 64 threads (on a 128 core system).
