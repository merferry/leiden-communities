In conclusion, this study addresses the optimization of Leiden method, a high-quality community detection algorithm, in the shared memory setting. We consider 9 different optimizations, which significantly improve the performance of the local-moving and the aggregation phases of the algorithm. On a server with dual 16-core Intel Xeon Gold 6226R processors, comparison with competitive open source implementations (Vite and Grappolo) and packages (NetworKit) show that our optimized implementation of Leiden, which we term as GVE-Leiden, is on average $50\times$, $22\times$, and $20\times$ faster than Vite, Grappolo, and NetworKit respectively. In addition, GVE-Leiden on average obtains $3.1\%$ higher modularity than Vite\ignore{(especially on web graph)}, and $0.6\%$ lower modularity than Grappolo and NetworKit\ignore{(especially on social networks with poor clustering)}. On a web graph with $3.8$ billion edges, GVE-Leiden identifies communities in $6.8$ seconds, and thus achieves a processing rate of $560$ million edges/s. In addition, GVE-Leiden achieves a strong scaling factor of $1.6\times$ for every doubling of threads. Looking ahead, future work could focus of designing dynamic algorithms for Leiden to accommodate dynamic graphs which evolve over time. This would contribute to interactive updation of community memberships of vertices in real-world scenarios.
