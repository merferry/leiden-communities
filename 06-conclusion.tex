In conclusion, this study addresses the design of the most optimized multicore implementation of the Leiden algorithm \cite{com-traag19}, to the best of our knowledge\ignore{a high-quality community detection algorithm that improves upon the popular Louvain method \cite{com-blondel08}, in the shared memory setting}.\ignore{We extend optimizations for the Louvain algorithm \cite{sahu2023gvelouvain} to our implementation of the Leiden algorithm, and use a greedy refinement phase where vertices greedily optimize for delta-modularity within their community bounds, which we observe, offers both better performance and quality than a randomized approach.} On a system equipped with two 16-core Intel Xeon Gold 6226R processors, our implementation of the Leiden algorithm, referred to as GVE-Leiden, attains a processing rate of $403 M$ edges per second on a $3.8 B$ edge graph. It surpasses the original Leiden implementation, igraph Leiden, and NetworKit Leiden by factors of $436\times$, $104\times$, and $8.2\times$ respectively. GVE-Leiden identifies communities of equivalent quality to the first two implementations, and $25\%$ higher quality than NetworKit. Doubling the number of threads results in an average performance scaling of $1.6\times$ for GVE-Leiden. In comparison to GVE-Louvain (our parallel Louvain implementation) \cite{sahu2023gvelouvain}, the original Leiden, igraph Leiden, and NetworKit Leiden, GVE-Leiden completely eliminates internally-disconnected communities.

In a previous version of this report, we implemented the refinement phase of the Leiden algorithm utilizing a \textit{constrained move} procedure, which does not guarantee the absence of disconnected communities. In this current version of the report, we have transitioned to employing the \textit{constrained merge} procedure alongside atomics to ensure no internally-disconnected communities. We also addressed issues in measuring disconnected communities for the original Leiden and igraph Leiden, which arose due to the number of vertices in a graph varying between the Matrix Market and the Edgelist formats (which does not have isolated vertices), and used the \texttt{RBConfigurationVertexPartition} with the original Leiden for large graphs (i.e., \textit{webbase-2001} and \textit{sk-2005}).
